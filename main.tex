% Modelo SBC exigido pelo edital
\documentclass{ppgpca-template}

\sloppy

\begin{document}

\titulo{Título do Anteprojeto (sem identificação do autor)}

\begin{resumo}
O presente anteprojeto, elaborado conforme o Edital de Seleção do PPGPCA/UFFS, apresenta o tema, objetivos, justificativa, metodologia e resultados esperados para uma pesquisa vinculada à linha de pesquisa selecionada pelo candidato. O texto segue o modelo SBC exigido e contempla de forma concisa a problemática, a relevância e a viabilidade da proposta, com aderência à área de Computação Aplicada.
\end{resumo}

\section{Introdução}

A introdução deve apresentar de forma clara o contexto da área, a problemática a ser enfrentada, os fundamentos teóricos iniciais e a importância da pesquisa no âmbito da Computação Aplicada. Deve também evidenciar a aderência do tema à linha de pesquisa selecionada no processo de inscrição, a saber: (i) Inteligência Artificial e Ciência de Dados ou (ii) Sistemas Computacionais Aplicados.

É fundamental apresentar o problema de pesquisa, ressaltando sua relevância científica, tecnológica e prática. Também deve ser descrita a motivação para a escolha do tema e a contribuição potencial que o estudo pode oferecer para o contexto profissional e/ou científico.

\section{Objetivos}

\subsection{Objetivo Geral}

Apresentar, de forma clara e abrangente, o propósito principal da pesquisa. O objetivo geral deve ser factível, relevante e diretamente relacionado ao problema delineado na introdução.

\subsection{Objetivos Específicos}

Os objetivos específicos devem decompor o objetivo geral em etapas práticas, observáveis e realizáveis dentro do período do mestrado profissional. Recomenda-se que:

\begin{itemize}
    \item sejam formulados de maneira clara e mensurável;
    \item descrevam etapas metodológicas necessárias;
    \item apresentem ações coerentes com a linha de pesquisa;
    \item sejam compatíveis com o cronograma previsto para o curso.
\end{itemize}

\section{Justificativa}

A justificativa deve evidenciar a relevância científica, técnica, social ou profissional da proposta. Devem ser apresentados argumentos que demonstrem a pertinência do tema e a contribuição potencial para a resolução de problemas reais, alinhada ao caráter profissional do Programa.

É importante indicar lacunas na literatura ou na prática profissional que sustentem a necessidade do estudo, demonstrando domínio inicial do assunto e coerência com a área de concentração do PPGPCA.

\section{Metodologia}

A metodologia deve descrever com precisão como a pesquisa será conduzida, incluindo:

\begin{itemize}
    \item abordagem metodológica (qualitativa, quantitativa, mista, estudo de caso, pesquisa experimental, etc.);
    \item técnicas e ferramentas de Computação a serem utilizadas (linguagens, bibliotecas, frameworks, ambientes de desenvolvimento);
    \item procedimentos de coleta, processamento e análise de dados;
    \item etapas previstas para modelagem, implementação, testes ou validações;
    \item formas de avaliação dos resultados, indicando métricas, indicadores ou critérios de desempenho.
\end{itemize}

A metodologia deve ser coerente com os objetivos e a justificativa, evidenciando a viabilidade das etapas dentro da duração do curso e dos recursos disponíveis.

\section{Resultados Esperados}

Nesta seção devem ser apresentados os produtos e contribuições previstos ao final do estudo, tais como:

\begin{itemize}
    \item desenvolvimento de modelos, sistemas, algoritmos, protótipos ou ferramentas;
    \item avanços teóricos ou metodológicos para a área de Computação Aplicada;
    \item impactos práticos no ambiente profissional ou institucional;
    \item potencial para geração de inovações ou melhoria de processos;
    \item possibilidade de publicações técnicas ou científicas.
\end{itemize}

Os resultados esperados devem ser compatíveis com a linha de pesquisa escolhida e com o caráter profissional do Programa.

\section{Cronograma de Execução}

Apresente uma previsão organizada das etapas da pesquisa ao longo do tempo, contemplando, por exemplo:

\begin{itemize}
    \item revisão bibliográfica;
    \item detalhamento da metodologia;
    \item desenvolvimento/implementação ou experimentação;
    \item análise e interpretação dos resultados;
    \item redação da dissertação;
    \item preparação de artigos técnicos ou científicos.
\end{itemize}

O cronograma deve ser factível, compatível com a carga horária do curso e com o prazo regulamentar para conclusão do mestrado profissional.

\section{Considerações Finais}

Esta seção deve sintetizar os principais aspectos apresentados ao longo do anteprojeto, destacando a relevância do problema investigado, a coerência entre objetivos, metodologia e resultados esperados, bem como a contribuição potencial da pesquisa para a área de Computação Aplicada. É recomendável que o texto reforce a viabilidade da proposta dentro do prazo do mestrado e evidencie o alinhamento com a linha de pesquisa escolhida. As considerações finais devem apresentar uma visão integrada do projeto, demonstrando clareza, consistência e maturidade na definição da proposta.

\section*{Instruções Gerais de Formatação}
\label{sec:formatacao}

Nota-se que esta seção contendo instruções de formatação e orientações técnicas serve apenas como referência para a elaboração do anteprojeto e deve ser removida da versão final submetida ao processo seletivo.

\subsection*{Instruções Gerais de Formatação}

O anteprojeto deve seguir o modelo SBC estabelecido no edital e não poderá ultrapassar cinco páginas de conteúdo, excluídas as páginas destinadas às referências bibliográficas. Os títulos das seções devem ser apresentados em \textbf{negrito}, tamanho 13pt, alinhados à esquerda, com espaçamento adicional de 12pt antes de cada título.

O primeiro parágrafo de cada seção não deve apresentar indentação. Os parágrafos subsequentes dentro da mesma seção devem iniciar com recuo de 1{,}27 cm na primeira linha, mantendo a consistência visual do texto.

As margens devem seguir o padrão configurado pelo template, e não devem ser incluídos cabeçalhos ou rodapés adicionais. O texto deve ser redigido em português, em linguagem acadêmica clara, objetiva e impessoal.

\subsection*{Seções e Parágrafos}

Os títulos das seções devem estar em \textbf{negrito}, tamanho 13pt, alinhados à esquerda, com espaçamento adicional de 12pt antes de cada título. A numeração das seções é opcional e pode ser utilizada conforme a preferência do autor, desde que mantida a consistência ao longo do documento.

O primeiro parágrafo de cada seção não deve ser indentado. Todos os parágrafos seguintes dentro da mesma seção devem iniciar com recuo de 1{,}27 cm na primeira linha, de modo a manter a padronização recomendada para artigos científicos em Computação.

\subsection*{Subseções}

Os títulos das subseções devem ser apresentados em \textbf{negrito}, tamanho 12pt, alinhados à esquerda. O estilo das subseções deve ser coerente com o das seções, garantindo uma hierarquia clara e facilmente identificável no documento.

\subsection*{Figuras e Legendas}\label{sec:figs}

As legendas de figuras e tabelas devem ser centralizadas quando ocuparem apenas uma linha (Figura~\ref{fig:exampleFig1}). Quando a legenda exceder uma linha, deve ser justificada e apresentar recuo de 0{,}8 cm em ambas as margens, como ilustrado na Figura~\ref{fig:exampleFig2}.

O texto das legendas deve utilizar fonte Helvetica, 10pt, em negrito, com 6 pontos de espaço antes e depois de cada legenda.

\begin{figure}[ht]
\centering
\includegraphics[width=.5\textwidth]{fig1.jpg}
\caption{Exemplo de figura simples}
\label{fig:exampleFig1}
\end{figure}

\begin{figure}[ht]
\centering
\includegraphics[width=.3\textwidth]{fig2.jpg}
\caption{Este é um exemplo de legenda de figura ocupando mais de uma linha e justificada considerando as margens mencionadas na Seção~\ref{sec:figs}.}
\label{fig:exampleFig2}
\end{figure}

No caso de tabelas, deve-se evitar o uso de fundos coloridos ou sombreados, bem como linhas espessas, duplas ou desnecessárias. Ao reportar dados empíricos, não utilize mais casas decimais do que as justificadas pela precisão e reprodutibilidade dos dados.

As legendas das tabelas devem ser posicionadas \textbf{antes} da tabela, conforme exemplo da Tabela~\ref{tab:exTable1}, utilizando fonte Helvetica, 10pt, em negrito, com 6 pontos de espaço antes e depois da legenda.

\begin{table}[ht]
\centering
\caption{Variáveis consideradas na avaliação de técnicas de interação}
  \begin{tabular*}{0.75\textwidth}{ccccc}
    \toprule
    \textbf{ID} & \textbf{\texttt{Profissão}}	 & \textbf{\texttt{Idade}}	& \textbf{\texttt{Proporção}} & \textbf{\texttt{Classe}} \\
    \midrule
      1   & indústria	& 34	& 2.96   & pago  \\
      2   & autônomo	& 41	& 4.64	 & atraso        \\
      3   & autônomo	& 36	& 3.22	 & atraso        \\
      4   & autônomo	& 41	& 3.11	 & atraso        \\
      5   & indústria	& 48	& 3.80	 & atraso        \\
      6   & indústria	& 61	& 2.52	 & pago  \\
      7   & autônomo	& 37	& 1.50	 & pago  \\
      8   & autônomo	& 40	& 1.93	 & pago  \\
      9   & indústria	& 33	& 5.25	 & atraso        \\
      10 & indústria	& 32	& 4.15	 & atraso        \\
      \bottomrule
  \end{tabular*}
\label{tab:exTable1}
\end{table}

\subsection*{Imagens}

Todas as imagens e ilustrações devem ser apresentadas preferencialmente em preto e branco ou em tons de cinza, sobretudo quando o documento for impresso. Para versões eletrônicas, pode-se utilizar cores, desde que não prejudiquem a legibilidade.

Para versões impressas, recomenda-se resolução de aproximadamente 600 dpi para imagens em preto e branco e entre 150 e 300 dpi para imagens em escala de cinza. Evite utilizar imagens com resolução excessiva, que podem causar lentidão na impressão sem ganho perceptível na qualidade visual.

\subsection*{Citações e Referências no Texto}

As citações devem ser realizadas utilizando o comando padrão \verb|\cite|, conforme práticas da SBC. Alguns exemplos:

\begin{itemize}
    \item Citação simples: \verb|\cite{knuth:84}|;
    \item Citações múltiplas: \verb|\cite{turing1936a,EWD:EWD215pub}|. 
    \item Citação em meio ao texto: \verb|\citeonline{EWD:EWD215pub}|
\end{itemize}

No corpo do texto, as citações aparecerão entre colchetes, como em \cite{knuth:84} e \cite{turing1936a,EWD:EWD215pub}, ou para citações em meio ao texto aparecerão como \citeonline{shannon1948mathematical}\footnote{Como usado em ``De acordo com \citeonline{shannon1948mathematical} ... ''}. Para  Recomenda-se citar apenas trabalhos efetivamente utilizados na fundamentação teórica, na discussão da metodologia ou na análise dos resultados.

A lista de referências deve seguir o estilo definido pelo pacote \texttt{abntex2cite}, sendo automaticamente formatada pelos comandos:

\begin{verbatim}
\bibliography{bibliografia}
\end{verbatim}

A primeira linha de cada referência não deve ter recuo, enquanto as subsequentes devem ser indentadas em aproximadamente 0,5 cm.

As referências bibliográficas devem ser claras, consistentes e seguir o estilo SBC. Utilize o ambiente de bibliografia configurado no arquivo \texttt{.bib} associado ao projeto, incluindo apenas obras que foram efetivamente citadas no texto.

\bibliography{bibliogafia}

\end{document}
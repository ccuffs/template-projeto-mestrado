% Modelo SBC exigido pelo edital
\documentclass{ppgpca-template}
\usepackage{graphicx,url}
\usepackage[utf8]{inputenc}

\sloppy

\begin{document}

\titulo{Título do Anteprojeto (sem identificação do autor)}

\begin{resumo}
O presente anteprojeto, elaborado conforme o Edital de Seleção do PPGPCA/UFFS, apresenta o tema, objetivos, justificativa, metodologia e resultados esperados para uma pesquisa vinculada à linha de pesquisa selecionada pelo candidato. O texto segue o modelo SBC, conforme exigência do edital, e contempla de forma concisa a problemática, a relevância e a viabilidade da proposta, com aderência à área de Computação Aplicada.
\end{resumo}

\section{Introdução}

A introdução deve apresentar de forma clara o contexto da área, a problemática a ser enfrentada, os fundamentos teóricos iniciais e a importância da pesquisa no âmbito da Computação Aplicada. Deve também evidenciar a aderência do tema à linha de pesquisa selecionada no processo de inscrição, a saber: (i) Inteligência Artificial e Ciência de Dados ou (ii) Sistemas Computacionais Aplicados, conforme descrito no edital.

É fundamental apresentar o problema de pesquisa, ressaltando sua relevância científica, tecnológica e prática. Também deve ser apresentada a motivação que leva à escolha do tema e a contribuição potencial que o estudo pode oferecer.

\section{Objetivos}

\subsection{Objetivo Geral}
Descrever, de forma clara e abrangente, o propósito principal da pesquisa. O objetivo geral deve ser factível, relevante e diretamente relacionado ao problema delineado na introdução.

\subsection{Objetivos Específicos}
Os objetivos específicos devem decompor o objetivo geral em etapas práticas, observáveis e realizáveis dentro do período do mestrado profissional. Exemplos de características desejáveis incluem:
\begin{itemize}
    \item Estabelecer metas mensuráveis;
    \item Descrever etapas metodológicas necessárias;
    \item Apresentar ações coerentes com a linha de pesquisa;
    \item Ser compatíveis com o cronograma previsto.
\end{itemize}

\section{Justificativa}

A justificativa deve evidenciar a relevância científica, técnica, social ou profissional da proposta. É necessário apresentar argumentos que demonstrem a pertinência do tema e a contribuição potencial para a resolução de problemas reais, alinhada ao caráter profissional do Programa.

Também deve ser explicitada a limitação existente na literatura ou na prática, destacando por que o presente estudo se torna necessário. O candidato deve demonstrar domínio inicial do assunto, conforme exigido nos critérios de avaliação do edital.

\section{Metodologia}

A metodologia deve descrever com precisão como a pesquisa será conduzida, incluindo:

\begin{itemize}
    \item método científico ou abordagem metodológica;
    \item técnicas e ferramentas de Computação a serem utilizadas;
    \item procedimentos de coleta, processamento e análise de dados;
    \item etapas previstas para implementação, testes ou validações;
    \item descrição da arquitetura, modelo, algoritmo ou sistema a ser desenvolvido;
    \item avaliação dos resultados e métricas pertinentes.
\end{itemize}

A metodologia deve ser coerente com os objetivos e justificativa. É necessário evidenciar a viabilidade das etapas dentro da duração do curso.

\section{Resultados Esperados}

Esta seção deve apresentar os produtos e contribuições previstos ao final do estudo, tais como:

\begin{itemize}
    \item desenvolvimento de modelos, sistemas, algoritmos ou protótipos;
    \item avanços teóricos ou metodológicos;
    \item impactos práticos no ambiente profissional;
    \item potenciais melhorias ou inovações tecnológicas;
    \item possibilidade de publicações ou comunicação científica.
\end{itemize}

Os resultados esperados devem ser compatíveis com a área de Computação Aplicada e alinhados aos objetivos específicos propostos.

\section{Cronograma de Execução}

Apresente uma previsão organizada das etapas da pesquisa ao longo do tempo, contemplando:

\begin{itemize}
    \item revisão bibliográfica;
    \item desenvolvimento da metodologia;
    \item implementação ou experimentação;
    \item análise e interpretação dos resultados;
    \item redação da dissertação;
    \item preparação de artigos técnicos ou científicos.
\end{itemize}

O cronograma deve ser factível e coerente com o prazo regulamentar do mestrado profissional.

\section{Referências}

As referências devem seguir o estilo SBC, conforme o pacote utilizado. É obrigatório citar apenas obras mencionadas no texto, com foco em literatura relevante que fundamente o problema e a metodologia.

\bibliographystyle{sbc}
\bibliography{sbc-template}

\end{document}